\chapter{الگوریتم بهینه‌سازی}
در این فصل ابتدا سیستم مدل و چنل مدل را بیان میکنیم و سپس تکنیک ساده‌کردن مسئله به کمک متغیرهای واسط را شرح میدهیم. در ادامه، الگوریتم اصلی حل مسئله یعنی بهینه‌سازی دوره‌ای را مرور میکنیم و سپس دوگانه لاگرانژ و الگوریتم جستجوی ریشه دوبخشی را توضیح مختصری میدهیم و به کمک آنها، ضرایب آنتن را بهینه میکنیم. سپس ضرایب فاز را بکمک روش معروف گرادیان نزولی بهینه‌سازی کرده و در نهایت، 3 ناحیه اصلی بهینه‌سازی ضرایب فاز را شرح میدهیم. 
\newpage
%%%%%%%%%%%%%%%%%%%%%%%%%%%%%%%%%%%%%%%%%%%
\section{
	مدل سیستم و مدل کانال
}

\subsection{مدل سیستم}
طبق سناریویی که در فصل قبل شرح دادیم، سیگنال دریافتی در هر کاربر به شرح زیر می‌باشد:
\begin{equation}
	x = \sum_{k=1}^{K} w_k s_k, \label{eq:transmitted_signal}
\end{equation}
که دیتای کاربر \lr{k} ام توسط $s_k$ نمایش داده میشود. همچنین فرض میشود که این $s_k$ ها به ازای ($k = 1, \ldots, K$) متغیرهای رندوم و مستقل با میانگین صفر و واریانس 1 باشند.\\
از طرفی $w_k \in \mathbb{C}^{N_t \times 1}$ بردار ضرایب آنتن فرستنده هستند.\\
اکنون سیگنال دریافتی برای کاربر \lr{k} ام به شرح زیر قابل ساده‌سازی میباشد:
\begin{align*}
	y_k = &\underbrace{h_{d,k}^H x}_
	{\substack{\text{لینک مستقیم}}}
	\quad + \quad \\
	&\underbrace{h_{r_1,k}^H \Phi_1 G_1 x}_
	{\substack{\text{بازتاب مرتبه اول سطح هوشمند 1}}}
	\quad + \quad
	\underbrace{h_{r_2,k}^H \Phi_2 G_2 x}_
	{\substack{\text{بازتاب مرتبه اول سطح هوشمند 2}}}
	\quad + \quad \\ 
	&\underbrace{h_{r_1,k}^H \Phi_1 D \Phi_2 G_2 x}_
	{\substack{\text{بازتاب مرتبه دوم سطح هوشمند}}}
	\quad + \quad
	\underbrace{h_{r_2,k}^H \Phi_2 D^H \Phi_1 G_1 x}_
	{\substack{\text{بازتاب مرتبه دوم سطح هوشمند}}} \quad + \quad
	\underbrace{u_k}_
	{\substack{\text{نویز سفید گاوسی جمع‌شونده}}}
\end{align*}

بگونه‌ای که $u_k \sim \mathcal{CN}(0, \sigma_0^2)$ نشان‌دهنده نویز گاوسی در کاربر \lr{k} ام میباشد.
\newpage
اکنون به سراغ نوشتن نسبت سیگنال به نویز بعلاوه تداخل(\lr{SINR}) میرویم.\\
برای هر کاربر، نویز گاوسی بعلاوه سیگنال کاربران دیگر بعنوان سیگنال مزاحم تلقی شده و در مخرج کسر قرار میگیرند. پس \lr{SINR} کاربر \lr{k} به شکل زیر است:
\[
\gamma_k = \frac{{\left|\left(h_{d,k}^H + h_{r_1,k}^H \Phi_1 G_1 + h_{r_2,k}^H \Phi_2 G_2 + h_{r_1,k}^H \Phi_1 D \Phi_2 G_2 + h_{r_2,k}^H \Phi_2 D^H \Phi_1 G_1 \right)w_k\right|^2}}{{\sum_{i=1,i\neq k}^{K} \left|\left(h_{d,k}^H + h_{r_1,k}^H \Phi_1 G_1 + h_{r_2,k}^H \Phi_2 G_2 + h_{r_1,k}^H \Phi_1 D \Phi_2 G_2 + h_{r_2,k}^H \Phi_2 D^H \Phi_1 G_1 \right)w_i\right|^2 + \sigma^2_0}}
\]
همچنین شرط توان نیز برای مجموع کاربران به‌شکل زیر نوشته میشود:
\[
\sum_{k=1}^{K} ||w_k||^2 \leq P_T
\]
بگونه‌ای که:
$\mathbf{W} = [\mathbf{w}_1, \mathbf{w}_2, \ldots, \mathbf{w}_K] \in \mathbb{C}^{N_t \times K}$

\subsection{مدل کانال}
برای مدل کردن کانال، ابتدا از مدل‌های رندوم استفاده مینماییم اما در صورت جواب گرفتن از الگوریتم، آن‌را برای مدل‌های رایلی و رایسی نیز بهینه‌سازی میکنیم.

\subsection{مسئله بهینه‌سازی}
در این پروژه، هدف، بیشینه‌کردن مجموع نرخ دریافتی کاربران است که به آن \lr{WSR} گفته میشود. برای اینکار میبایست ضرایب آنتن و ضرایب فاز سطوح هوشمند را بصورت همزمان و توام بهینه‌سازی نماییم زیرا این دو دسته از متغیرها بر یکدیگر تاثیر میگذارند و نمیتوانند بصورت جداگانه بهینه‌سازی شوند. همچنین باید هر جوابی که برای ضرایب آنتن ارائه میشود، در شرط توان صدق نماید. پس مسئله را میتوان به شکل زیر فرموله کرد:
\begin{align*}
	(P1) \quad \max_{\mathbf{W}, \boldsymbol{\Phi_1}, \boldsymbol{\Phi_2}} \quad & f_1(\mathbf{W}, \boldsymbol{\Phi_1}, \boldsymbol{\Phi_2}) = \sum_{k=1}^{K} \omega_k \log_2(1 + \gamma_k) \\
	\text{به شرط} \quad & \theta_{m_k} \in \mathcal{F}, \quad \forall m_k = 1, \ldots, M_k, \tag{4-2} \\
	& \sum_{k=1}^{K} \| \mathbf{w}_k \|_2^2 \leq P_T,
\end{align*}

%%%%%%%%%%%%%%%%%%%%%%%%%%%%%%%%%%%%%%%%%%%
\section{}
%%%%%%%%%%%%%%%%%%%%%%%%%%%%%%%%%%%%%%%%%%%
\section{}
%%%%%%%%%%%%%%%%%%%%%%%%%%%%%%%%%%%%%%%%%%%
\section{}
%%%%%%%%%%%%%%%%%%%%%%%%%%%%%%%%%%%%%%%%%%%
\section{}
%%%%%%%%%%%%%%%%%%%%%%%%%%%%%%%%%%%%%%%%%%%
\section{}
%%%%%%%%%%%%%%%%%%%%%%%%%%%%%%%%%%%%%%%%%%%
\section{}
%%%%%%%%%%%%%%%%%%%%%%%%%%%%%%%%%%%%%%%%%%%
\section{}
%%%%%%%%%%%%%%%%%%%%%%%%%%%%%%%%%%%%%%%%%%%
\newpage
‌