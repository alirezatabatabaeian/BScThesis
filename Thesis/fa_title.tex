%% -!TEX root = AUTthesis.tex
% در این فایل، عنوان پایان‌نامه، مشخصات خود، متن تقدیمی‌، ستایش، سپاس‌گزاری و چکیده پایان‌نامه را به فارسی، وارد کنید.
% توجه داشته باشید که جدول حاوی مشخصات پروژه/پایان‌نامه/رساله و همچنین، مشخصات داخل آن، به طور خودکار، درج می‌شود.
%%%%%%%%%%%%%%%%%%%%%%%%%%%%%%%%%%%%
% دانشکده، آموزشکده و یا پژوهشکده  خود را وارد کنید
\faculty{دانشکده مهندسی برق}
% گرایش و گروه آموزشی خود را وارد کنید
\department{گرایش مخابرات}
% عنوان پایان‌نامه را وارد کنید
\fatitle{بیشینه‌سازی عملکرد سیستم 
	\\[.35 cm]
	چند ورودی-تک خروجی به کمک 
	\\[.35 cm]
	سطوح بازتاب دهنده هوشمند 
	\\[.35 cm]
	با استفاده از بهینه‌سازی کلاسیک
}
% نام استاد(ان) راهنما را وارد کنید
\firstsupervisor{دکتر محمدجواد عمادی}
%\secondsupervisor{استاد راهنمای دوم}
% نام استاد(دان) مشاور را وارد کنید. چنانچه استاد مشاور ندارید، دستور پایین را غیرفعال کنید.
%\firstadvisor{نام کامل استاد مشاور}
%\secondadvisor{استاد مشاور دوم}
% نام نویسنده را وارد کنید
\name{سید علیرضا }
% نام خانوادگی نویسنده را وارد کنید
\surname{طباطبائیان نیم‌آوردی}
%%%%%%%%%%%%%%%%%%%%%%%%%%%%%%%%%%
\thesisdate{شهریور 1402}

% چکیده پایان‌نامه را وارد کنید
\fa-abstract{
این پروژه به بررسی کاربرد‌های سطوح بازتابنده هوشمند (\lr{IRS}) در زمینه شبکه‌های ارتباطی آینده، به ویژه در \lr{6G}، می‌پردازد. هدف این مطالعه بهینه‌سازی تابع هدف مجموع نرخ وزن‌دار برای یک سناریو واقعی با دو کاربر است. در این سناریو، دو واحد \lr{IRS} در نظر گرفته شده است و بازتاب بین دو سطح هوشمند نیز لحاظ شده است. همچنین در جلوی یکی از کاربران یک مانع قرار دارد. برای بهبود عملکرد شبکه‌های ارتباطی، رویکرد بهینه‌سازی مشترک به‌کار گرفته می‌شود که هم ضرایب \lr{IRS} و هم ضرایب آنتن را بهینه می‌کند. این بهینه‌سازی با استفاده از الگوریتم \lr{CD} یا همان \lr{AO} صورت می‌گیرد که درون آن یک \lr{Gradient Descent} نیز وجود دارد و با حلقه‌ای تکراری، ضرایب تابش را بهبود می‌بخشد تا عملکرد کلی سیستم را بیشینه کند. این تحقیق همچنین ارزیابی نسبت سیگنال به تداخل بعلاوه نویز (\lr{SINR}) در حضور بازتاب بین دو \lr{IRS} را بررسی میکند. فرآیند بهینه‌سازی شامل چندین مرحله کلیدی است. ابتدا، تبدیل دوگان لاگرانژی به کار گرفته می‌شود تا تابع لگاریتمی حذف شود و مسئله بهینه‌سازی ساده شود. به علاوه، تکنیک‌های برنامه‌ریزی کسری برای تبدیل عبارات کسری به توابع خطی به کار گرفته می‌شوند که فرآیند بهینه‌سازی را ساده‌تر می‌کند.
در فصل اول، مروری کلی بر مفاهیم اولیه مخابرات خواهیم داشت.
در فصل دوم، مفاهیم کلی بهینه‌سازی را در حد مباحث مقدماتی یاد خواهیم‌گرفت.
در فصل سوم، سناریو مسئله و کارهای شبیه به سناریو را بررسی میکنیم.
در فصل چهار، سیستم مدل و الگوریتم حل مسئله را بررسی میکنیم.
در نهایت در فصل پنجم نیز نتایج حاصل از شبیه‌سازی را بررسی و مقایسه خواهیم کرد.
}

% کلمات کلیدی پایان‌نامه را وارد کنید
\keywords{
	سطوح بازتاب‌دهنده هوشمند، بازتاب بین سطحی، مجموع نرخ داده‌ها، نزول مختصاتی، گرادیان نزولی
}



\AUTtitle
%%%%%%%%%%%%%%%%%%%%%%%%%%%%%%%%%%
\vspace*{7cm}
\thispagestyle{empty}
\begin{center}
\includegraphics[height=5cm,width=12cm]{besm}
\end{center}