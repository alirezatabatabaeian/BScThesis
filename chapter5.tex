\chapter{نتایج و پیشنهادات}
در این فصل ابتدا نتایج حاصل از پیاده‌سازی الگوریتم توضیح داده‌شده در فصل قبل را برای هر دو سناریو مورد بررسی قرار میدهیم و سپس به دو مورد از کارهایی که در آینده برای پیشبرد این تحقیق و کمک به سایر محققان قابل انجام است اشاره میکنیم.
\newpage
%%%%%%%%%%%%%%%%%%%%%%%%%%%%%%%%%%%%%%%%%%%
\section{نتایج}

\subsection{نتایج و تحلیل بهینه‌سازی بدون مانع}

\subsection{نتایج و تحلیل بهینه‌سازی با مانع}

%%%%%%%%%%%%%%%%%%%%%%%%%%%%%%%%%%%%%%%%%%%
\section{پیشنهادات}

در این بخش به بررسی دو پیشنهاد(کارهای آینده) میپردازیم که میتواند توجه زیادی را به خود جلب کرده و مشکلات زیادی را حل کند:

\subsection{بازتاب مرتبه 3}

همانطور که دراین ستاریو بررسی شد، بازتاب بین سطوح هوشمند فقط تا بازتاب مرتبه 1 لحاظ شده است و در بخش نتایج مشاهده شد که این بازتاب، تاثیر چندانی بر روی بهینه شدن خروجی به نسبت پیچیدگی که به مسئله اضافه میکند، ندارد اما گاهی ناچار هستیم که این بازتاب را لحاظ کنیم. مثلا در هنگامی که دید مستقیم به فرستنده و یکی از دو سطوح هوشمند وجود ندارد، ناچار به لحاظ این بازتاب هستیم. 

اکنون یکی از مسائلی که مطرح میشود، این است که اگر 3 سطح هوشمند استفاده کنیم و دید مستقیم به دو سطح نداشته باشیم، آیا از نظر  نرخ دریافتی کاربر، بهینه است که از سیگنال بازتابی مرتبه 2 استفاده کنیم تا اطلاعات را به کاربر برسانیم؟

این یکی از سوالاتی است که میتواند در آینده و به کمک سایر محققان جواب داده شود. 

\subsection{ساخت فریم‌ورک برای بهینه‌سازی سطوح هوشمند}

یکی از چالش‌های اصلی که در برنامه‌نویسی این پروژه برای بهینه‌سازی ضرایب فاز وجود داشت، نبود هیچ فریم‌ورک یا همان چارچوب برنامه‌نویسی خاصی برای حل این مسائل است. ابتدا مجبور بودیم به دنبال الگوریتم بهینه بگردیم و سپس این الگوریتم را پیاده‌سازی نماییم و این امکان وجود داشت که الگوریتم مورد نظر در بیشینه‌های محلی غیر بهینه گیر کرده و نتواند به جواب قابل‌قبولی برسد. یا حتی ممکن بود این الگوریتم همگرا نشود که محقق در این صورت به ناچار مجبور است به سراغ سایر الگوریتم‌ها رفته که این کار اصلا از نظر زمان و انرژی، بهینه نیست.

اکنون میتوان به عنوان یکی از کارهای آینده، فریم‌ورکی طراحی کرد که کاربر، سناریو خود را به آن داده و این فریم‌ورک، انواع روش‌های مرسوم بهینه‌سازی در این حوزه را امتحان کرده و بهینه‌ترین جواب را انتخاب و به ما نمایش دهد.

البته که این مورد نیاز به صرف زمان و انرژی زیادی دارد که صرفا در قالب پروژه‌ای صنعتی و با وجود سرمایه‌گذار قابل انجام است.
%%%%%%%%%%%%%%%%%%%%%%%%%%%%%%%%%%%%%%%%%%%
\newpage
‌