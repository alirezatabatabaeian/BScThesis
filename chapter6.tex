\chapter{علم مخابرات}
در این فصل ...کامل شود...
%%%%%%%%%%%%%%%%%%%%%%%%%%%%%%%%%%%%%%%%%%%
\section{مقدمه‌ای بر علم مخابرات}
علم مخابرات یک رشته مهم و چندگانه‌ای است که به مطالعه انتقال، تبادل و تفسیر اطلاعات بین افراد، سیستم‌ها و دستگاه‌ها می‌پردازد. این علم به دنبال بهبود کارایی و امنیت انتقال اطلاعات در فرآیندهای مختلف می‌باشد. مخابرات تأثیر بسزایی بر توسعه اجتماعی، اقتصادی و فناوری دارد و در زمینه‌های مختلفی نظیر تلفن، اینترنت، تلویزیون و ارتباطات نظامی به کار می‌رود.

\subsection{
	اصول اساسی مخابرات
}
\begin{enumerate}
	\item \textbf{
		انتقال اطلاعات:
	}
	 عملیات انتقال داده‌ها و اطلاعات از یک مکان به مکان دیگر از اصول اساسی مخابرات است. این انتقال می‌تواند به صورت سیمی (مانند کابل‌ها) یا بی‌سیم (از طریق امواج رادیویی یا مایکروویو) انجام شود.
	
	\item \textbf{مدولاسیون و دمدولاسیون:} برای انتقال اطلاعات، آن‌ها به سیگنال‌هایی تبدیل می‌شوند که به راحتی قابل انتقال باشند. این فرآیند به نام مدولاسیون شناخته می‌شود. در مقابل، دمدولاسیون فرآیند بازیابی اطلاعات از سیگنال‌های مدوله شده را شامل می‌شود.
	
	\item \textbf{کدگذاری و کدگشایی:} برای افزایش امنیت و کاهش تداخلات، اطلاعات ممکن است با استفاده از روش‌های کدگذاری به یک فرمت خاص تبدیل شوند. کدگشایی نیز فرآیند بازگرداندن اطلاعات به فرمت اصلی را توصیف می‌کند.
\end{enumerate}

\section{کاربردها و گستره علم مخابرات}
\subsection{کاربردهای اصلی مخابرات}
\begin{enumerate}
	\item \textbf{تلفنی و تصویری:} انتقال صدا و تصاویر در عصر حاضر از طریق تلفن و ویدئوکنفرانس از کاربردهای اصلی مخابرات به شمار می‌آید.
	
	\item \textbf{شبکه‌های کامپیوتری:} اینترنت و شبکه‌های دیگر از راه‌های ارتباطی بر اساس اصول مخابراتی هستند که به ما امکان ارسال و دریافت اطلاعات را از سراسر جهان می‌دهند.
	
	\item \textbf{تلویزیون و رادیو:} انتقال برنامه‌های تلویزیونی و رادیویی به تلویزیون‌ها و رادیوها نیز از طریق اصول مخابراتی انجام می‌شود.
	
	\item \textbf{تله‌مدیسین و تله‌پزشکی:} از مخابرات برای انتقال اطلاعات پزشکی و تصاویر پزشکی به منظور تشخیص بیماری‌ها و مشاوره‌ی پزشکی از راه دور استفاده می‌شود.
	
	\item \textbf{ارتباطات فضایی:} در ماموریت‌های فضایی و ارتباطات با ماهواره‌ها، اصول مخابراتی برای ارسال و دریافت اطلاعات به‌کار می‌روند.
	
	\item \textbf{شبکه‌های اجتماعی:} ارتباطات اجتماعی آنلاین از طریق شبکه‌های اجتماعی نیز از تکنیک‌های مخابراتی برای انتقال اطلاعات استفاده می‌کنند.
\end{enumerate}

\subsection{گستره علم مخابرات}
علم مخابرات به وسیع‌ترین معانی به سایر حوزه‌های علمی نیز ارتباط دارد. به عنوان مثال:
\begin{itemize}
	\item \textbf{مخابرات نظامی:} در ارتباطات نظامی، امنیت، ردیابی، جاسوسی و انتقال داده‌ها در شرایط خاص مورد بررسی قرار می‌گیرد.
	
	\item \textbf{پزشکی:} از مخابرات در تله‌مدیسین، انتقال تصاویر پزشکی و اطلاعات بیماری‌ها برای تشخیص از راه دور استفاده می‌شود.
	
	\item \textbf{حمل و نقل:} ارتباطات در خودروها، قطارها و هواپیماها جهت بهبود امنیت و کارایی حمل و نقل مورد استفاده قرار می‌گیرد.
	
	\item \textbf{تکنولوژی اطلاعات و ارتباطات (\lr{ICT}):} علم مخابرات به‌عنوان پایه‌ای از علوم مرتبط با \lr{ICT}، به توسعه ابزارها، سیستم‌ها و نرم‌افزارهای ارتباطی کمک می‌کند.
	
	\item \textbf{امنیت اطلاعات:} در دنیای امروز، امنیت اطلاعات و محرمانگی داده‌ها نقش بسزایی دارد که از اصول مخابراتی برای رمزنگاری و حفاظت در برابر نفوذ استفاده می‌شود.
	
	\item \textbf{شبکه‌های هوش مصنوعی:} انتقال داده‌ها و اطلاعات در شبکه‌های هوش مصنوعی و اینترنت اشیاء نیز از اصول مخابراتی بهره می‌برد.
\end{itemize}

علم مخابرات به دلیل تأثیرات وسیع‌تری که بر ابزارها، فرآیندها و جوامع دارد، به یکی از پایه‌های اصلی توسعه فناوری و ارتباطات در جوامع مدرن تبدیل شده است. این علم به دنبال بهبود ارتباطات بین انسان‌ها و دستگاه‌ها در سراسر جهان است و در عصر اطلاعات، نقش بسزایی دارد.

%%%%%%%%%%%%%%%%%%%%%%%%%%%%%%%%%%%%%%%%%%%
\section{آغاز و پیشینه تاریخی علم مخابرات}

در طول تاریخ، انسان‌ها همواره تلاش کرده‌اند تا ارتباطات خود را بهبود بخشند. از ارسال پیغام‌های ساده با کمک آتش یا دیگر علائم تا به اختراع وسایل پیام‌رسان پیشرفته، مراحل مختلفی در تاریخچه مخابرات وجود دارد.

\subsection{سازوکارهای اولیه}
در دوران‌های اولیه تاریخ، ارتباطات انسان‌ها از طریق نمادها، علائم و صداها انجام می‌شد. انسان‌ها از طریق آتش و دود، نشانه‌های راهبردی می‌ساختند که از دور قابل مشاهده بودند. همچنین، استفاده از پیغام‌های صوتی با استفاده از زنگ‌ها و دستگاه‌های ساده دیگر از مکانیسم‌های اولیه مخابرات بود.

\subsection{استفاده از حیوانات و افراد}
با گذر زمان، انسان‌ها از حیوانات و افراد برای انتقال پیغام‌ها و اطلاعات به فاصله‌های بیشتر استفاده کردند. از پیغام‌رسانی با استفاده از کوفی‌ها و مسیریابی پیاده‌روی‌ها تا ایجاد سیستم‌هایی برای انتقال پیام‌ها با استفاده از اسب‌ها، مثال‌هایی از این دوران‌ها هستند.

\subsection{اختراع رادیو}
با پیشرفت فناوری در قرن 19، اختراع رادیو توسط علمایی چون گوگلیلمو مارکونی و نیکولا تسلا انقلابی در زمینه مخابرات ایجاد کرد. رادیو به انسان‌ها امکان ارسال و دریافت امواج الکترومغناطیسی را به صورت بی‌سیم فراهم کرد و ارتباطات بی‌سیم را آغاز کرد.

\section{تکنولوژی مخابرات در قرن بیست و یکم}
در قرن بیست و یکم، پیشرفت‌های فراوانی در زمینه علم مخابرات رخ داد. با ظهور کامپیوترها و توسعه اینترنت، ارتباطات به طور جهانی و پیچیده‌تری انجام می‌شود. فناوری‌های بی‌سیم مانند موبایل، وای‌فای، بلوتوث و ماهواره‌ها ارتباطات را به سطح جدیدی رسانده‌اند.

\subsection{انقلاب دیجیتالی و اینترنت}
در دهه‌های اخیر، انقلاب دیجیتالی و ظهور اینترنت تغییرات اساسی در مخابرات ایجاد کرده‌اند. اینترنت به عنوان یک شبکه جهانی، میلیاردها دستگاه را به یکدیگر متصل کرده و به اشتراک گذاری اطلاعات، ارتباطات اجتماعی و تجارت را تغییر داده است.

\subsection{شبکه‌های اجتماعی و ارتباطات اجتماعی آنلاین}
با ظهور شبکه‌های اجتماعی مانند فیسبوک، توییتر، اینستاگرام و لینکدین، ارتباطات اجتماعی به صورت آنلاین و از راه دور انجام می‌شود. این شبکه‌ها نه تنها به اشتراک گذاری تجربیات و اطلاعات، بلکه در پیدا کردن کار، تبلیغات و تأثیرگذاری نیز نقش دارند.

\subsection{مخابرات \lr{5G} و پیشرفت‌های آینده}
تکنولوژی مخابرات همچنان در حال پیشرفت است. به عنوان مثال، فناوری \lr{5G} با امکانات بالاتری در سرعت انتقال داده، کاهش تأخیر و افزایش توانایی اتصال بهتر، در حال توسعه است و قرار است تاثیرات چشمگیری بر ارتباطات و صنایع داشته باشد.

\section*{
	مخابرات سیار یا مخابرات بیسیم
}

\subsection*{
	مفهوم و معنی
}
مخابرات سیار یا مخابرات بیسیم به انتقال اطلاعات و ارتباطات بین دستگاه‌ها از طریق امواج رادیویی یا وسایل بی‌سیم مشغول است. این فناوری به ما این امکان را می‌دهد که در هر مکانی و در هر زمانی ارتباط داشته باشیم، بدون نیاز به سیم‌کشی یا اتصال فیزیکی مستقیم.

\subsection{تاریخچه مخابرات بیسیم}
مخابرات بیسیم از زمان اختراع رادیو تا به امروز تغییرات بزرگی را تجربه کرده است. اختراع تلگراف بیسیم توسط مارکونی در اواخر قرن نوزدهم توسط وایرلس تلگراف راه‌اندازی شد. پس از آن، با اختراع رادیو و سایر فناوری‌های بی‌سیم، مخابرات بیسیم به شکلی کاملاً جدید تبدیل شد.

\subsection{انواع مخابرات بیسیم}
مخابرات بیسیم به انواع مختلفی تقسیم می‌شود. از جمله انواع مخابرات بیسیم می‌توان به مخابرات سلولی، وای‌فای، بلوتوث، نسل‌های مختلف تلفن همراه مانند \lr{3G}, \lr{4G}
 و \lr{5G}
 و ارتباطات ماهواره‌ای اشاره کرد.

\subsection{کاربردهای مخابرات بیسیم}
مخابرات بیسیم در زندگی روزمره ما نقش بزرگی ایفا می‌کند. از تماس‌های تلفنی و پیامک‌ها تا استفاده از اینترنت بی‌سیم، تلویزیون‌های هوشمند، دستگاه‌های هوشمند، سامانه‌های ردیابی موقعیت جغرافیایی، سیستم‌های اطلاع‌رسانی اضطراری و بسیاری از فناوری‌های دیگر، مخابرات بیسیم به‌طور گسترده در حیات ما حضور دارد.

\section{
	مخابرات سلولی
}

\subsection{تعریف}
مخابرات سلولی، یا شبکه‌های تلفن همراه، سیستم‌های ارتباطی بی‌سیم هستند که از امواج رادیویی برای انتقال صدا، داده و اطلاعات استفاده می‌کنند. این سیستم‌ها به دستگاه‌های تلفن همراه اجازه می‌دهند تا به تبادل اطلاعات با یکدیگر و به شبکه ارتباطی متصل شوند.

\subsection{تقسیمات شبکه‌های سلولی}
شبکه‌های تلفن همراه به چندین منطقه تقسیم می‌شوند که به این مناطق سلول‌ گفته می‌شود. هر سلول یک محدوده جغرافیایی را پوشش می‌دهد و دارای یک تجهیزات ارتباطی مرکزی است که به عنوان ترانس‌هدایت‌کننده مرکزی (\lr{BTS}) شناخته می‌شود.

\subsection{تکنولوژی‌های نسل‌های مختلف تلفن همراه}
شبکه‌های تلفن همراه در طول زمان به نسل‌های مختلفی تقسیم می‌شوند که به توانایی‌های خاص خود معروف هستند. از نسل اول تا نسل پنجم، هر نسل به سرعت انتقال داده، پهنای باند، قابلیت‌های صوتی و تصویری و کاربردهای دیگر ارتباطات بی‌سیم تاثیر می‌گذارد.

\subsection{تغییرات اجتماعی و اقتصادی از طریق مخابرات سلولی}
مخابرات سلولی تغییرات عمده‌ای در جوامع و اقتصادها به وجود آورده است. از تجارت الکترونیک و کسب‌وکارهای آنلاین گرفته تا ارتباطات اجتماعی و تغییرات در رفتارهای انسانی، این تکنولوژی اثرات چشمگیری را در سطح جامعه داشته است.
