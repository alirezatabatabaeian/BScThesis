\chapter{علم مخابرات}
در این فصل ابتدا علم مخابرات را تعریف و اصول اساسی آن را بررسی میکنیم. سپس به کاربردها و گستره این علم میپردازیم. در ادامه با بیان تاریخچه آن، درکی نسبی از این علم پیدا میکنیم. سپس بر بخش خاصی از مخابرات به‌نام مخابرات بیسیم یا مخابرات سیار و سپس مخابرات سلولی متمرکز میشویم. در نهایت نیز انواع فرستنده-گیرنده را از حیث تعداد آنتن بررسی کرده و مفهوم کلی سطوح بازتاب‌دهنده هوشمند و کاربردهای آن را مورد بررسی قرار میدهیم.
\newpage
%%%%%%%%%%%%%%%%%%%%%%%%%%%%%%%%%%%%%%%%%%%
\section{مقدمه‌ای بر علم مخابرات}
علم مخابرات یک رشته مهم و چندگانه‌ای است که به مطالعه انتقال، تبادل و تفسیر اطلاعات بین افراد، سیستم‌ها و دستگاه‌ها می‌پردازد. این علم به دنبال بهبود کارایی و امنیت انتقال اطلاعات در فرآیندهای مختلف می‌باشد. مخابرات تأثیر بسزایی بر توسعه اجتماعی، اقتصادی و فناوری دارد و در زمینه‌های مختلفی نظیر تلفن، اینترنت، تلویزیون و ارتباطات نظامی به کار می‌رود.

\subsection{
	اصول اساسی مخابرات
}
\begin{enumerate}
	\item \textbf{
		انتقال اطلاعات:
	}
	 عملیات انتقال داده‌ها و اطلاعات از یک مکان به مکان دیگر از اصول اساسی مخابرات است. این انتقال می‌تواند به صورت سیمی (مانند کابل‌ها) یا بی‌سیم (از طریق امواج رادیویی یا مایکروویو) انجام شود.
	
	\item \textbf{مدولاسیون و دمدولاسیون:} برای انتقال اطلاعات، آن‌ها به سیگنال‌هایی تبدیل می‌شوند که به راحتی قابل انتقال باشند. این فرآیند به نام مدولاسیون شناخته می‌شود. در مقابل، دمدولاسیون فرآیند بازیابی اطلاعات از سیگنال‌های مدوله شده را شامل می‌شود.
	
	\item \textbf{کدگذاری و کدگشایی:} برای افزایش امنیت و کاهش تداخلات، اطلاعات ممکن است با استفاده از روش‌های کدگذاری به یک فرمت خاص تبدیل شوند. کدگشایی نیز فرآیند بازگرداندن اطلاعات به فرمت اصلی را توصیف می‌کند.
\end{enumerate}

\section{کاربردها و گستره علم مخابرات}
\subsection{کاربردهای اصلی مخابرات}
\begin{enumerate}
	\item \textbf{تلفنی و تصویری:} انتقال صدا و تصاویر در عصر حاضر از طریق تلفن و ویدئوکنفرانس از کاربردهای اصلی مخابرات به شمار می‌آید.
	
	\item \textbf{شبکه‌های کامپیوتری:} اینترنت و شبکه‌های دیگر از راه‌های ارتباطی بر اساس اصول مخابراتی هستند که به ما امکان ارسال و دریافت اطلاعات را از سراسر جهان می‌دهند.
	
	\item \textbf{تلویزیون و رادیو:} انتقال برنامه‌های تلویزیونی و رادیویی به تلویزیون‌ها و رادیوها نیز از طریق اصول مخابراتی انجام می‌شود.
	
	\item \textbf{تله‌مدیسین و تله‌پزشکی:} از مخابرات برای انتقال اطلاعات پزشکی و تصاویر پزشکی به منظور تشخیص بیماری‌ها و مشاوره‌ی پزشکی از راه دور استفاده می‌شود.
	
	\item \textbf{ارتباطات فضایی:} در ماموریت‌های فضایی و ارتباطات با ماهواره‌ها، اصول مخابراتی برای ارسال و دریافت اطلاعات به‌کار می‌روند.
	
	\item \textbf{شبکه‌های اجتماعی:} ارتباطات اجتماعی آنلاین از طریق شبکه‌های اجتماعی نیز از تکنیک‌های مخابراتی برای انتقال اطلاعات استفاده می‌کنند.
\end{enumerate}

\subsection{گستره علم مخابرات}
علم مخابرات به وسیع‌ترین معانی به سایر حوزه‌های علمی نیز ارتباط دارد. به عنوان مثال:
\begin{itemize}
	\item \textbf{مخابرات نظامی:} در ارتباطات نظامی، امنیت، ردیابی، جاسوسی و انتقال داده‌ها در شرایط خاص مورد بررسی قرار می‌گیرد.
	
	\item \textbf{پزشکی:} از مخابرات در تله‌مدیسین، انتقال تصاویر پزشکی و اطلاعات بیماری‌ها برای تشخیص از راه دور استفاده می‌شود.
	
	\item \textbf{حمل و نقل:} ارتباطات در خودروها، قطارها و هواپیماها جهت بهبود امنیت و کارایی حمل و نقل مورد استفاده قرار می‌گیرد.
	
	\item \textbf{تکنولوژی اطلاعات و ارتباطات (\lr{ICT}):} علم مخابرات به‌عنوان پایه‌ای از علوم مرتبط با \lr{ICT}، به توسعه ابزارها، سیستم‌ها و نرم‌افزارهای ارتباطی کمک می‌کند.
	
	\item \textbf{امنیت اطلاعات:} در دنیای امروز، امنیت اطلاعات و محرمانگی داده‌ها نقش بسزایی دارد که از اصول مخابراتی برای رمزنگاری و حفاظت در برابر نفوذ استفاده می‌شود.
	
	\item \textbf{شبکه‌های هوش مصنوعی:} انتقال داده‌ها و اطلاعات در شبکه‌های هوش مصنوعی و اینترنت اشیاء نیز از اصول مخابراتی بهره می‌برد.
\end{itemize}

علم مخابرات به دلیل تأثیرات وسیع‌تری که بر ابزارها، فرآیندها و جوامع دارد، به یکی از پایه‌های اصلی توسعه فناوری و ارتباطات در جوامع مدرن تبدیل شده است. این علم به دنبال بهبود ارتباطات بین انسان‌ها و دستگاه‌ها در سراسر جهان است و در عصر اطلاعات، نقش بسزایی دارد.
%%%%%%%%%%%%%%%%%%%%%%%%%%%%%%%%%%%%%%%%%%%
\section{آغاز و پیشینه تاریخی علم مخابرات}

در طول تاریخ، انسان‌ها همواره تلاش کرده‌اند تا ارتباطات خود را بهبود بخشند. از ارسال پیغام‌های ساده با کمک آتش یا دیگر علائم تا به اختراع وسایل پیام‌رسان پیشرفته، مراحل مختلفی در تاریخچه مخابرات وجود دارد.

\subsection{سازوکارهای اولیه}
در دوران‌های اولیه تاریخ، ارتباطات انسان‌ها از طریق نمادها، علائم و صداها انجام می‌شد. انسان‌ها از طریق آتش و دود، نشانه‌های راهبردی می‌ساختند که از دور قابل مشاهده بودند. همچنین، استفاده از پیغام‌های صوتی با استفاده از زنگ‌ها و دستگاه‌های ساده دیگر از مکانیسم‌های اولیه مخابرات بود.

\subsection{استفاده از حیوانات و افراد}
با گذر زمان، انسان‌ها از حیوانات و افراد برای انتقال پیغام‌ها و اطلاعات به فاصله‌های بیشتر استفاده کردند. از پیغام‌رسانی با استفاده از کوفی‌ها و مسیریابی پیاده‌روی‌ها تا ایجاد سیستم‌هایی برای انتقال پیام‌ها با استفاده از اسب‌ها، مثال‌هایی از این دوران‌ها هستند.

\subsection{اختراع رادیو}
با پیشرفت فناوری در قرن 19، اختراع رادیو توسط علمایی چون گوگلیلمو مارکونی و نیکولا تسلا انقلابی در زمینه مخابرات ایجاد کرد. رادیو به انسان‌ها امکان ارسال و دریافت امواج الکترومغناطیسی را به صورت بی‌سیم فراهم کرد و ارتباطات بی‌سیم را آغاز کرد.
%%%%%%%%%%%%%%%%%%%%%%%%%%%%%%%%%%%%%%%%%%%
\section{تکنولوژی مخابرات در قرن بیست و یکم}
در قرن بیست و یکم، پیشرفت‌های فراوانی در زمینه علم مخابرات رخ داد. با ظهور کامپیوترها و توسعه اینترنت، ارتباطات به طور جهانی و پیچیده‌تری انجام می‌شود. فناوری‌های بی‌سیم مانند موبایل، وای‌فای، بلوتوث و ماهواره‌ها ارتباطات را به سطح جدیدی رسانده‌اند.

\subsection{انقلاب دیجیتالی و اینترنت}
در دهه‌های اخیر، انقلاب دیجیتالی و ظهور اینترنت تغییرات اساسی در مخابرات ایجاد کرده‌اند. اینترنت به عنوان یک شبکه جهانی، میلیاردها دستگاه را به یکدیگر متصل کرده و به اشتراک گذاری اطلاعات، ارتباطات اجتماعی و تجارت را تغییر داده است.

\subsection{شبکه‌های اجتماعی و ارتباطات اجتماعی آنلاین}
با ظهور شبکه‌های اجتماعی مانند فیسبوک، توییتر، اینستاگرام و لینکدین، ارتباطات اجتماعی به صورت آنلاین و از راه دور انجام می‌شود. این شبکه‌ها نه تنها به اشتراک گذاری تجربیات و اطلاعات، بلکه در پیدا کردن کار، تبلیغات و تأثیرگذاری نیز نقش دارند.

\subsection{مخابرات \lr{5G} و پیشرفت‌های آینده}
تکنولوژی مخابرات همچنان در حال پیشرفت است. به عنوان مثال، فناوری \lr{5G} با امکانات بالاتری در سرعت انتقال داده، کاهش تأخیر و افزایش توانایی اتصال بهتر، در حال توسعه است و قرار است تاثیرات چشمگیری بر ارتباطات و صنایع داشته باشد.
%%%%%%%%%%%%%%%%%%%%%%%%%%%%%%%%%%%%%%%%%%%
\section{
	مخابرات سیار یا مخابرات بیسیم
}

\subsection{
	تعریف
}
مخابرات سیار یا مخابرات بیسیم به انتقال اطلاعات و ارتباطات بین دستگاه‌ها از طریق امواج رادیویی یا وسایل بی‌سیم مشغول است. این فناوری به ما این امکان را می‌دهد که در هر مکانی و در هر زمانی ارتباط داشته باشیم، بدون نیاز به سیم‌کشی یا اتصال فیزیکی مستقیم.

\subsection{تاریخچه مخابرات بیسیم}
مخابرات بیسیم از زمان اختراع رادیو تا به امروز تغییرات بزرگی را تجربه کرده است. اختراع تلگراف بیسیم توسط مارکونی در اواخر قرن نوزدهم توسط وایرلس تلگراف راه‌اندازی شد. پس از آن، با اختراع رادیو و سایر فناوری‌های بی‌سیم، مخابرات بیسیم به شکلی کاملاً جدید تبدیل شد.

\subsection{انواع مخابرات بیسیم}
مخابرات بیسیم به انواع مختلفی تقسیم می‌شود. از جمله انواع مخابرات بیسیم می‌توان به مخابرات سلولی، وای‌فای، بلوتوث، نسل‌های مختلف تلفن همراه مانند \lr{3G}, \lr{4G}
 و \lr{5G}
 و ارتباطات ماهواره‌ای اشاره کرد.

\subsection{کاربردهای مخابرات بیسیم}
مخابرات بیسیم در زندگی روزمره ما نقش بزرگی ایفا می‌کند. از تماس‌های تلفنی و پیامک‌ها تا استفاده از اینترنت بی‌سیم، تلویزیون‌های هوشمند، دستگاه‌های هوشمند، سامانه‌های ردیابی موقعیت جغرافیایی، سیستم‌های اطلاع‌رسانی اضطراری و بسیاری از فناوری‌های دیگر، مخابرات بیسیم به‌طور گسترده در حیات ما حضور دارد.
%%%%%%%%%%%%%%%%%%%%%%%%%%%%%%%%%%%%%%%%%%%
\section{
	مخابرات سلولی
}

\subsection{تعریف}
مخابرات سلولی، یا شبکه‌های تلفن همراه، سیستم‌های ارتباطی بی‌سیم هستند که از امواج رادیویی برای انتقال صدا، داده و اطلاعات استفاده می‌کنند. این سیستم‌ها به دستگاه‌های تلفن همراه اجازه می‌دهند تا به تبادل اطلاعات با یکدیگر و به شبکه ارتباطی متصل شوند.

\subsection{تقسیمات شبکه‌های سلولی}
شبکه‌های تلفن همراه به چندین منطقه تقسیم می‌شوند که به این مناطق سلول‌ گفته می‌شود. هر سلول یک محدوده جغرافیایی را پوشش می‌دهد و دارای یک تجهیزات ارتباطی مرکزی است که به عنوان ترانس‌هدایت‌کننده مرکزی (\lr{BTS}) شناخته می‌شود.

\subsection{تکنولوژی‌های نسل‌های مختلف تلفن همراه}
شبکه‌های تلفن همراه در طول زمان به نسل‌های مختلفی تقسیم می‌شوند که به توانایی‌های خاص خود معروف هستند. از نسل اول تا نسل پنجم، هر نسل به سرعت انتقال داده، پهنای باند، قابلیت‌های صوتی و تصویری و کاربردهای دیگر ارتباطات بی‌سیم تاثیر می‌گذارد.

\subsection{تغییرات اجتماعی و اقتصادی از طریق مخابرات سلولی}
مخابرات سلولی تغییرات عمده‌ای در جوامع و اقتصادها به وجود آورده است. از تجارت الکترونیک و کسب‌وکارهای آنلاین گرفته تا ارتباطات اجتماعی و تغییرات در رفتارهای انسانی، این تکنولوژی اثرات چشمگیری را در سطح جامعه داشته است.
%%%%%%%%%%%%%%%%%%%%%%%%%%%%%%%%%%%%%%%%%%%
\section{انواع فرستنده-گیرنده از نظر تعداد آنتن}

\subsection{
	ارتباط \lr{SISO} : تک ورودی - تک خروجی
}

ارتباط \lr{SISO}، مخفف \lr{Single-Input Single-Output}، یک سیستم ارتباطی بی‌سیم ابتدایی است که در آن یک فرستنده تنها از یک آنتن برای ارسال سیگنال به یک گیرنده استفاده می‌کند. در این تنظیم، یک آنتن در فرستنده و یک آنتن در گیرنده وجود دارد. ارتباط در جهت یکطرفه است، به این معنی که فرستنده اطلاعات را ارسال کرده، و گیرنده آن را دریافت می‌کند.

سیستم‌های تک ورودی - تک خروجی در کاربردهای مختلفی از جمله پخش رادیویی سنتی، دستگاه‌های واکی‌تاکی و بسیاری از سیستم‌های ارتباطی بی‌سیم اولیه استفاده می‌شوند. با وجود سادگی، سیستم‌های تک ورودی - تک خروجی محدودیت‌هایی در زمینه نرخ داده، محدوده پوشش و قابلیت اطمینان دارند. تداخل، خنثی شدن و نویز ممکن است بر کیفیت سیگنال دریافتی نیز تأثیر بگذارد.

\subsection{
	ارتباط \lr{MISO} : چند ورودی - تک خروجی
}

ارتباط \lr{MISO}، مخفف \lr{Multiple-Input Single-Output}، در شرایطی به کار می‌رود که یک گیرنده تنها سیگنال‌ها را از چند فرستنده متفاوت که هرکدام دارای آنتن خود هستند، دریافت می‌کند. این تنظیم برای بهبود کیفیت سیگنال، پوشش و نرخ داده مورد استفاده قرار می‌گیرد. چندین فرستنده می‌توانند با همکاری به تقویت قدرت سیگنال دریافتی و کاهش تأثیر خنثی شدن و خنثی شدگی کمک کنند.

سیستم‌های \lr{MISO} اغلب در شبکه‌های سلولی به کار می‌روند، جایی که چندین ایستگاه پایه سیگنال‌ها را به یک دستگاه تلفن همراه ارسال می‌کنند. دستگاه تلفن همراه از طریق ترکیب سیگنال‌ها از آنتن‌های مختلف به کیفیت سیگنال دریافتی بهتری دست می‌یابد. این تنظیم به حل مشکلاتی مانند خنثی شدن چندمسیره و اثر سایه ای کمک می‌کند و باعث بهبود کیفیت ارتباط می‌شود.

\subsection{
		ارتباط \lr{SIMO} : تک ورودی - چند خروجی
}

ارتباط \lr{SIMO}، مخفف \lr{Single-Input Multiple-Output}، به سیستمی اشاره دارد که یک فرستنده تنها سیگنال را به چندین گیرنده ارسال می‌کند، هر کدام دارای آنتن خود هستند. این تنظیم امکان دریافت تنوعی را فراهم می‌کند، به طوری که گیرنده‌های چندگانه سیگنال را از مسیرهای مختلف دریافت می‌کنند. این کار باعث مقابله با محوشوندگی و بهبود قابلیت اطمینان ارتباط می‌شود.

سیستم‌های \lr{SIMO} در مواقعی مورد استفاده قرار می‌گیرند که ارتباط قابل اعتماد بسیار مهم است، مانند ارتباطات بی‌سیم در محیط‌های چالشی یا پوشش دهی داخلی. با استفاده از چندین آنتن در سمت گیرنده، تأثیرات انتشار چندمسیره و محوشوندگی می‌تواند کاهش یابد و از این رو کیفیت ارتباط بهبود می‌یابد.

\subsection{
	ارتباط \lr{MIMO} : چند ورودی - چند خروجی
}

ارتباط \lr{MIMO}، مخفف \lr{Multiple-Input Multiple-Output}، یک سیستم پیشرفته ارتباطی بی‌سیم است که همزمان از چندین آنتن در فرستنده و گیرنده استفاده می‌کند. تکنولوژی \lr{MIMO} از تنوع فضایی و انتشار چندمسیره برای دستیابی به نرخ داده، پوشش و اطمینان بیشتر استفاده می‌کند. چندین جریان داده می‌توانند به صورت همزمان روی همان فرکانس ارسال شوند که باعث افزایش ظرفیت سیستم می‌شود.

سیستم‌های \lr{MIMO} به عنوان یک تکنولوژی بنیادی در استانداردهای ارتباطی بی‌سیم مدرن، از جمله \lr{Wi-Fi} و شبکه‌های سلولی (مانند \lr{4G} و \lr{5G}) به کار می‌روند. با استفاده از چندین آنتن، سیستم‌های \lr{MIMO} می‌توانند از تنوع فضایی برای بهبود کیفیت سیگنال، مهار تداخل و دستیابی به نرخ داده بالاتر استفاده کنند.
%%%%%%%%%%%%%%%%%%%%%%%%%%%%%%%%%%%%%%%%%%%
\section{
	سطوح بازتاب دهنده هوشمند (\lr{IRS})
}

\subsection{
	مفهوم کلی
}

سطوح بازتاب دهنده هوشمند همچنین با نام‌های سطوح بازتاب‌کننده قابل تنظیم یا سطوح بازتاب دهنده هوشیار، فناوری پیشرفته‌ای در ارتباطات بی‌سیم و پردازش سیگنال هستند. این ایده شامل استفاده از سطوحی با عناصر غیرفعال(\lr{passive}) نمایش داده می‌شود که برای کنترل و مدیریت امواج الکترومغناطیسی به کار می‌روند. با تغییر بازتاب و پراکندگی سیگنال‌ها، سطور بازتاب‌دهنده هوشمند می‌تواند کیفیت سیگنال‌ها را افزایش داده، پوشش را افزایش داده و عملکرد کلی سیستم‌های ارتباطی بی‌سیم را بهبود بخشد. در ادامه، مروری اجمالی از این سطوح و کاربردهای آن آورده شده است:

\subsection{
	اصول کار
}

\lr{IRS} شامل یک سطح با عناصر غیرفعال کوچک مانند آنتن‌ها، سوئیچ‌ها یا مواد قابل تنظیم می‌شود. این عناصر می‌توانند فاز و مقدار موج‌های الکترومغناطیسی ورودی را تغییر دهند. با تنظیم دقیق فاز و مقدار، \lr{IRS} می‌تواند سیگنال‌ها را در جهات مورد نظر بازتاب کند تا تداخل سازنده ایجاد کرده و قدرت سیگنال در گیرنده را بهبود دهد.

\subsection{کاربردهای سطوح هوشمند}

- \textbf{
	افزایش کیفیت ارتباط بی‌سیم:
}
 یکی از کاربردهای اصلی سطوح هوشمند، بهبود کیفیت و پوشش سیستم‌های ارتباطی بی‌سیم است. با تغییر سیگنال‌های بازتابی، می‌توان اثر تداخل‌های چندمسیره را کاهش داده، نسبت سیگنال به نویز را بهبود داده و پوشش را به مناطقی با سیگنال ضعیف ترتیب دهد.

- \textbf{
	پوشش داخلی و نفوذ:
}
 \lr{IRS} می‌تواند در محیط‌های داخلی برای رفع مشکلات نفوذ سیگنال به دلیل دیوارها، مبلمان و موانع دیگر استفاده شود. این فناوری می‌تواند نفوذ سیگنال و پوشش را بهبود داده و ارتباطات داخلی را قابل اعتماد‌تر کند.

- \textbf{
	\lr{5G} و به بعد:
} \lr{IRS}
 در شبکه‌های \lr{5G} و سیستم‌های ارتباطی آینده توانمندی‌های بسیاری دارد. این وسیله می‌تواند به زیرساخت‌های سلولی موجود اضافه شود تا نرخ داده را بهبود داده، تاخیر را کاهش داده و ظرفیت شبکه را افزایش دهد.

- \textbf{کارآیی انرژی:} \lr{IRS} نیز می‌تواند به بهبود کیفیت سیگنال‌های دریافتی کمک کند و به دستگاه‌ها اجازه دهد با توان کمتر عمل کنند که این منجر به صرفه‌جویی در مصرف انرژی می‌شود.

- \textbf{ارتباطات امن:} با کنترل جهت سیگنال و کاهش نفوذ سیگنال به مناطق غیر مجاز می‌تواند به افزایش امنیت ارتباط کمک کند.

- \textbf{اینترنت اشیا (\lr{IoT}):} در سناریوهای اینترنت اشیا که تعداد زیادی دستگاه با یکدیگر ارتباط دارند، سطوح هوشمند می‌تواند به بهبود کارآیی و قابلیت اعتماد شبکه کمک کند.

- \textbf{ترکیب \lr{Massive MIMO}:} \lr{IRS} می‌تواند با سیستم‌های \lr{MIMO} بزرگ ترکیب شود تا عملکرد آن‌ها را بهبود دهد و سیستم ترکیبی با اجزای فعال و غیرفعال ایجاد کند.

- \textbf{شکل‌دهی به شعاع و جهت:} \lr{IRS} با شکل‌دهی به شعاع و جهت دینامیک اجازه می‌دهد تا با تطابق به شرایط مختلف ارتباطات پیش بروید.

\subsection{چالش‌ها و چشم‌اندازهای آینده}

با وجود مزایای فراوان، مواردی مانند پیاده‌سازی عملی، همگام‌سازی و تخمین کانال در مورد \lr{IRS} چالش‌هایی وجود دارد. با این حال، تحقیقات و توسعه‌های جاری به منظور غلبه بر این موانع و بهره‌برداری کامل از تکنولوژی سطوح هوشمند در حال انجام است.

\subsection{نتیجه‌گیری}

سطوح بازتاب دهنده هوشمند نشان‌دهنده یک فناوری تحول‌آفرین در ارتباطات بی‌سیم هستند. توانایی این فناوری در تغییر موج‌های الکترومغناطیسی، امکانات جدیدی را برای بهبود کیفیت سیگنال‌ها، پوشش و ظرفیت ارتباطات ارائه می‌دهد. در حالی که تحقیقات و نوآوری‌ها ادامه دارند، انتظار می‌رود که این سطوح نقش اساسی را در تدوین آینده سیستم‌های ارتباطات بی‌سیم ایفا کند.

\newpage
‌